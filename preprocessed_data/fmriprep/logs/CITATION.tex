\PassOptionsToPackage{unicode=true}{hyperref} % options for packages loaded elsewhere
\PassOptionsToPackage{hyphens}{url}
%
\documentclass[]{article}
\usepackage{lmodern}
\usepackage{amssymb,amsmath}
\usepackage{ifxetex,ifluatex}
\usepackage{fixltx2e} % provides \textsubscript
\ifnum 0\ifxetex 1\fi\ifluatex 1\fi=0 % if pdftex
  \usepackage[T1]{fontenc}
  \usepackage[utf8]{inputenc}
  \usepackage{textcomp} % provides euro and other symbols
\else % if luatex or xelatex
  \usepackage{unicode-math}
  \defaultfontfeatures{Ligatures=TeX,Scale=MatchLowercase}
\fi
% use upquote if available, for straight quotes in verbatim environments
\IfFileExists{upquote.sty}{\usepackage{upquote}}{}
% use microtype if available
\IfFileExists{microtype.sty}{%
\usepackage[]{microtype}
\UseMicrotypeSet[protrusion]{basicmath} % disable protrusion for tt fonts
}{}
\IfFileExists{parskip.sty}{%
\usepackage{parskip}
}{% else
\setlength{\parindent}{0pt}
\setlength{\parskip}{6pt plus 2pt minus 1pt}
}
\usepackage{hyperref}
\hypersetup{
            pdfborder={0 0 0},
            breaklinks=true}
\urlstyle{same}  % don't use monospace font for urls
\setlength{\emergencystretch}{3em}  % prevent overfull lines
\providecommand{\tightlist}{%
  \setlength{\itemsep}{0pt}\setlength{\parskip}{0pt}}
\setcounter{secnumdepth}{0}
% Redefines (sub)paragraphs to behave more like sections
\ifx\paragraph\undefined\else
\let\oldparagraph\paragraph
\renewcommand{\paragraph}[1]{\oldparagraph{#1}\mbox{}}
\fi
\ifx\subparagraph\undefined\else
\let\oldsubparagraph\subparagraph
\renewcommand{\subparagraph}[1]{\oldsubparagraph{#1}\mbox{}}
\fi

% set default figure placement to htbp
\makeatletter
\def\fps@figure{htbp}
\makeatother

\usepackage[]{natbib}
\bibliographystyle{plainnat}

\date{}

\begin{document}

Results included in this manuscript come from preprocessing performed
using \emph{fMRIPrep} 1.3.2 (\citet{fmriprep1}; \citet{fmriprep2};
RRID:SCR\_016216), which is based on \emph{Nipype} 1.1.9
(\citet{nipype1}; \citet{nipype2}; RRID:SCR\_002502).

\begin{description}
\item[Anatomical data preprocessing]
The T1-weighted (T1w) image was corrected for intensity non-uniformity
(INU) with \texttt{N4BiasFieldCorrection} \citep{n4}, distributed with
ANTs 2.2.0 \citep[RRID:SCR\_004757]{ants}, and used as T1w-reference
throughout the workflow. The T1w-reference was then skull-stripped with
a \emph{Nipype} implementation of the \texttt{antsBrainExtraction.sh}
workflow (from ANTs), using OASIS30ANTs as target template. Brain
surfaces were reconstructed using \texttt{recon-all} \citep[FreeSurfer
6.0.1, RRID:SCR\_001847,][]{fs_reconall}, and the brain mask estimated
previously was refined with a custom variation of the method to
reconcile ANTs-derived and FreeSurfer-derived segmentations of the
cortical gray-matter of Mindboggle
\citep[RRID:SCR\_002438,][]{mindboggle}. Spatial normalization to the
\emph{ICBM 152 Nonlinear Asymmetrical template version 2009c}
\citep[RRID:SCR\_008796]{mni152nlin2009casym} was performed through
nonlinear registration with \texttt{antsRegistration} (ANTs 2.2.0),
using brain-extracted versions of both T1w volume and template. Brain
tissue segmentation of cerebrospinal fluid (CSF), white-matter (WM) and
gray-matter (GM) was performed on the brain-extracted T1w using
\texttt{fast} \citep[FSL 5.0.9, RRID:SCR\_002823,][]{fsl_fast}.
\item[Functional data preprocessing]
For each of the 6 BOLD runs found per subject (across all tasks and
sessions), the following preprocessing was performed. First, a reference
volume and its skull-stripped version were generated using a custom
methodology of \emph{fMRIPrep}. The BOLD reference was then
co-registered to the T1w reference using \texttt{bbregister}
(FreeSurfer) which implements boundary-based registration \citep{bbr}.
Co-registration was configured with nine degrees of freedom to account
for distortions remaining in the BOLD reference. Head-motion parameters
with respect to the BOLD reference (transformation matrices, and six
corresponding rotation and translation parameters) are estimated before
any spatiotemporal filtering using \texttt{mcflirt} \citep[FSL
5.0.9,][]{mcflirt}. BOLD runs were slice-time corrected using
\texttt{3dTshift} from AFNI 20160207 \citep[RRID:SCR\_005927]{afni}. The
BOLD time-series, were resampled to surfaces on the following spaces:
\emph{fsaverage5}. The BOLD time-series (including slice-timing
correction when applied) were resampled onto their original, native
space by applying a single, composite transform to correct for
head-motion and susceptibility distortions. These resampled BOLD
time-series will be referred to as \emph{preprocessed BOLD in original
space}, or just \emph{preprocessed BOLD}. The BOLD time-series were
resampled to MNI152NLin2009cAsym standard space, generating a
\emph{preprocessed BOLD run in MNI152NLin2009cAsym space}. First, a
reference volume and its skull-stripped version were generated using a
custom methodology of \emph{fMRIPrep}. Several confounding time-series
were calculated based on the \emph{preprocessed BOLD}: framewise
displacement (FD), DVARS and three region-wise global signals. FD and
DVARS are calculated for each functional run, both using their
implementations in \emph{Nipype} \citep[following the definitions
by][]{power_fd_dvars}. The three global signals are extracted within the
CSF, the WM, and the whole-brain masks. Additionally, a set of
physiological regressors were extracted to allow for component-based
noise correction \citep[\emph{CompCor},][]{compcor}. Principal
components are estimated after high-pass filtering the
\emph{preprocessed BOLD} time-series (using a discrete cosine filter
with 128s cut-off) for the two \emph{CompCor} variants: temporal
(tCompCor) and anatomical (aCompCor). Six tCompCor components are then
calculated from the top 5\% variable voxels within a mask covering the
subcortical regions. This subcortical mask is obtained by heavily
eroding the brain mask, which ensures it does not include cortical GM
regions. For aCompCor, six components are calculated within the
intersection of the aforementioned mask and the union of CSF and WM
masks calculated in T1w space, after their projection to the native
space of each functional run (using the inverse BOLD-to-T1w
transformation). The head-motion estimates calculated in the correction
step were also placed within the corresponding confounds file. All
resamplings can be performed with \emph{a single interpolation step} by
composing all the pertinent transformations (i.e.~head-motion transform
matrices, susceptibility distortion correction when available, and
co-registrations to anatomical and template spaces). Gridded
(volumetric) resamplings were performed using
\texttt{antsApplyTransforms} (ANTs), configured with Lanczos
interpolation to minimize the smoothing effects of other kernels
\citep{lanczos}. Non-gridded (surface) resamplings were performed using
\texttt{mri\_vol2surf} (FreeSurfer).
\end{description}

Many internal operations of \emph{fMRIPrep} use \emph{Nilearn} 0.5.0
\citep[RRID:SCR\_001362]{nilearn}, mostly within the functional
processing workflow. For more details of the pipeline, see
\href{https://fmriprep.readthedocs.io/en/latest/workflows.html}{the
section corresponding to workflows in \emph{fMRIPrep}'s documentation}.

\hypertarget{references}{%
\subsubsection{References}\label{references}}

\bibliography{/usr/local/miniconda/lib/python3.7/site-packages/fmriprep/data/boilerplate.bib}

\end{document}
